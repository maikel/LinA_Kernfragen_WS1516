\documentclass{scrartcl}

% Sprache und Schrift
\usepackage[utf8]{inputenc}
\usepackage[ngerman]{babel}
\usepackage{amsthm}
\usepackage{libertine} \usepackage[libertine]{newtxmath}
% \usepackage{amsmath, amssymb}
\usepackage{multicol}
% \usepackage{mathspec}

% Mathematik
% \usepackage{amsthm}

% bessere Listen
\usepackage{paralist}

% eigene Umgebungen
\theoremstyle{definition}
\newtheorem{frage}{Frage}[section]

% mathematische Aliase
\newcommand{\R}{\mathbb R}
\newcommand{\N}{\mathbb N}
\newcommand{\C}{\mathbb C}
\newcommand{\Q}{\mathbb Q}
\newcommand{\Z}{\mathbb Z}
\newcommand{\sprod}[2]{\left\langle #1, #2 \right\rangle}
\newcommand{\mat}[1]{\begin{pmatrix} #1 \end{pmatrix}}
\renewcommand{\Im}{\mathrm{Im}}
\let\tmpphi\phi
\let\phi\varphi
\let\varphi\tmpphi
\DeclareMathOperator{\lin}{span}
\DeclareMathOperator{\Mat}{Mat}

\begin{document}
{\raggedleft Autor: M. Nadolski \hfill Wintersemester 2015/2016}\\
E-Mail: nadolski@fu-berlin.de

\begin{center}
\LARGE\textbf {Kernfragen}\\
\large ``Lineare Algebra fur Ingenieurwissenschaften''
\end{center}

\section {Vektorräume}

Im Folgenden sei $K$ ein beliebiger Körper und $V$ ein $K$-Vektorraum.

\begin{frage}
Was ist ein Körper?
\end{frage}

\begin{frage}
Wie ist ein $K$-Vektorraum $(V, +, \cdot)$ definiert?
\end{frage}

\begin{frage}
Wie sind die komplexen Zahlen $\C$ definiert? Wie sind $z_1 \cdot z_2$
und $z_1 + z_2$ für komplexe Zahlen $z_1, z_2 \in \C$ definiert.
\end{frage}

\begin{frage}
Sei $z \in \C$ und $\bar z$ die konjugiert komplexe Zahl zu $z$. Wie ist
$\bar z$ definiert? Bestimmte Real- und Imaginärteil folgender Ausdrücke:
\begin{multicols}{2}
\begin{enumerate}[(i)]
    \item $z \bar z$
    \item $\frac 12 ( z + \bar z )$
    \item $\frac 12 ( z - \bar z )$
    \item $z = \frac 1 i$
    \item $z = \frac 1 {1 - i \sqrt 3}$
    \item $z = \frac {(-2 + 5i) \cdot (1 + 3i)}{2 + 3i} 
             - \left( \frac 2 {13} - \frac 3 {13} i \right)$
    \item $z = e^{i \phi},\: \phi \in \R$
\end{enumerate}
\end{multicols}
\end{frage}

\begin{frage}
Sei $z \in \C$. Bestimme alle Lösungen von:
\begin{multicols}{2}
\begin{enumerate}[(i)]
    \item $z^2 = 1$
    \item $z^2 = -1$
    \item $z^2 + (1 + i) z + i = 0$
    \item $z^3 = -i$
    \item $z^4 = -4$
\end{enumerate}
\end{multicols}
\end{frage}

\begin{frage}
Wie sind Untervektorräume eines $K$-Vektorraums $V$ definiert?
\end{frage}

\begin{frage}
Sei $U \subset V$ eine Menge. Wie ist die lineare H"ulle $\lin U$ definiert?
\end{frage}

\begin{frage}
Sei $U \subset V$ eine Menge. Wann ist $U$ ein Erzeugendensystem von $V$?
\end{frage}

\begin{frage}
Seien $v_1, \ldots, v_n \in V$. Wann heißen sie linear unabhängig?
Sei $U \subset V$ eine beliebige Menge. Wann heißt $U$ linear unabhängig?
\end{frage}

\begin{frage}
Gebe zwei verschiedene aber äquivalente Definitionen einer Basis von $V$ 
an. Wie ist die Dimension $\dim_K V$ von $V$ definiert?
\end{frage}

\clearpage

\begin{frage}
Bestimme die Dimensionen folgender Vektorr"aume:
\begin{enumerate}[(i)]
    \item $\dim_\R \R^2$
    \item $\dim_\R \C$
    \item $\dim_\C \C$
    \item $\dim_\Q \R$
\end{enumerate}
\end{frage}

Von nun an sei $V$ ein endlich-dimensionaler Vektorraum

\begin{frage}
Seien $U,W$ Untervektorräume von $V$. Wie sind Summe $U + W$ und direkte
Summe $U \oplus W$ definiert.
\end{frage}

\begin{frage}
Beweise oder Widerlege: Ist $B_U$  eine Basis von $U$ und $B_W$ eine Basis 
von $W$, dann ist $B_U \cup B_W$ eine Basis von $U + W$.
\end{frage}

\begin{frage}
Bestimme die Dimension des Unterraums von $\R^4$, der von den Vektoren
\[
    \begin{pmatrix} 1 \\ -4 \\ -2 \\ 1 \end{pmatrix}, \:
    \begin{pmatrix} 1 \\ -3 \\ -1 \\ 2 \end{pmatrix}, \:
    \begin{pmatrix} 3 \\ -8 \\ -2 \\ 7 \end{pmatrix}
\]
aufgespannt wird.
\end{frage}

\begin{frage}
Gegeben Sei
\[
    B = \left\{
        \begin{pmatrix} 1 \\ 1 \\ 1 \end{pmatrix}, \:
        \begin{pmatrix} 1 \\ 1 \\ 0 \end{pmatrix}, \:
        \begin{pmatrix} 1 \\ 0 \\ 0 \end{pmatrix}
    \right\}
\]
Zeige, dass $B$ eine Basis des $\R^3$ ist!
\end{frage}

\begin{frage}
Gegeben Sei eine Menge $M$ von Vektoren des $\R^3$:
\[
    M = \left\{
        \begin{pmatrix} 1 \\ 2 \\ 3 \end{pmatrix}, \:
        \begin{pmatrix} 0 \\ 1 \\ 1 \end{pmatrix}, \:
        \begin{pmatrix} 2 \\ 5 \\ 7 \end{pmatrix}
    \right\}
\]
Welche Dimension hat $\lin M$? Gebe eine Basis von $\lin M$ an!
\end{frage}

\begin{frage}
Gegeben sei eine Menge $M$ von Vektoren des $\R^3$:
\[
    M = \left\{ 
        \begin{pmatrix} 1 \\ 2 \\ 3 \end{pmatrix}, \:
        \begin{pmatrix} 0 \\ 1 \\ 1 \end{pmatrix}
    \right\}
\]
Ergänze $M$ zu einer Basis von $\R^3$!
\end{frage}

\section {Matrizen und Gauß'sches Eliminationsverfahren}

Von hier an bezeichnen wir mit $\Mat(K, n, m)$ die Menge aller
$n \times m$-Matrizen mit Eintr"agen aus dem K"orper $K$.

\begin{frage}
Wie sind Matrizenaddition und -multiplikation definiert? Ist
$(\Mat(K, n, n), +, \cdot)$ ein Ring? Ist es ein K"orper? Begr"undung!
\end{frage}

\begin{frage}
Wie ist der Spalten- bzw. Zeilenrang einer Matrix $M \in \Mat(\R, n, m)$
definiert?
\end{frage}

\begin{frage}
Rechne folgende Produkte aus:
\begin{enumerate}[(i)]
\item \[
M_1 = 
\begin{pmatrix}
  1  & 0 & 2 \\
  2  & 1 & 5 \\
  3  & 1 & 7
\end{pmatrix} \cdot \begin{pmatrix}
  1 \\ 0 \\ 0
\end{pmatrix}, \qquad M_2 =
\begin{pmatrix}
  1  & 0 & 2 \\
  2  & 1 & 5 \\
  3  & 1 & 7
\end{pmatrix} \cdot \begin{pmatrix}
  0 \\ 1 \\ 0
\end{pmatrix} \qquad M_3 =
\begin{pmatrix}
  1  & 0 & 2 \\
  2  & 1 & 5 \\
  3  & 1 & 7
\end{pmatrix} \cdot \begin{pmatrix}
  0 \\ 0 \\ 1
\end{pmatrix}
\]

\item \[ M_1 =
\begin{pmatrix}
  1 & 0 & 0 \\
  0 & 1 & 0 \\
  0 & 0 & 1 
\end{pmatrix} \cdot \begin{pmatrix}
  1  & 0 & 2 \\
  2  & 1 & 5 \\
  3  & 1 & 7
\end{pmatrix}, \qquad M_2 =
\begin{pmatrix}
  1 & 0 & 0 \\
  0 & 0 & 1 \\
  0 & 1 & 0 
\end{pmatrix} \cdot \begin{pmatrix}
  1  & 0 & 2 \\
  2  & 1 & 5 \\
  3  & 1 & 7
\end{pmatrix}, \qquad M_3 =
\begin{pmatrix}
  1  & 0 & 2 \\
  2  & 1 & 5 \\
  3  & 1 & 7
\end{pmatrix} \cdot \begin{pmatrix}
  1 & 0 & 0 \\
  0 & 0 & 1 \\
  0 & 1 & 0 
\end{pmatrix}
\]
\item \[ M_1 =
\begin{pmatrix}
  1 & 0 & 0 \\
  1 & 1 & 0 \\
  0 & 0 & 1 
\end{pmatrix} \cdot \begin{pmatrix}
  1  & 0 & 2 \\
  2  & 1 & 5 \\
  3  & 1 & 7
\end{pmatrix} \qquad M_2 =
\begin{pmatrix}
  1 & 0 & 1 \\
  0 & 1 & 0 \\
  0 & 0 & 1 
\end{pmatrix} \cdot \begin{pmatrix}
  1  & 0 & 2 \\
  2  & 1 & 5 \\
  3  & 1 & 7
\end{pmatrix}, \qquad M_3 =
\begin{pmatrix}
  1  & 0 & 2 \\
  2  & 1 & 5 \\
  3  & 1 & 7
\end{pmatrix} \cdot \begin{pmatrix}
  1 & 0 & 1 \\
  0 & 1 & 0 \\
  0 & 0 & 1 
\end{pmatrix}
\]
\end{enumerate}
\end{frage}

\begin{frage}
Welche Typen von Elementarmatrizen gibt es? Was haben sie mit dem
Gauß-Algorithmus zu tun?
\end{frage}

\begin{frage}
Im Folgenden ist eine $3\times3$-Matrix gesucht, die bei der
Mutliplikation von links ...
\begin{enumerate}[(i)]
\item ... die erste und zweite Zeile vertauscht.
\item ... die dritte Zeile durch ein 3-faches ihrer selbst ersetzt.
\item ... auf die dritte Zeile die zweite Zeile addiert.
\item ... die Punkte (i)-(iii) in dieser Reihenfolge bewirkt.
\end{enumerate}
\end{frage}

\begin{frage}
Bestimme f"ur die Matrix
\[ M =
\begin{pmatrix}
  1  & 0 & 2 \\
  2  & 1 & 5 \\
  3  & 1 & 7
\end{pmatrix}
\] die Matrizen $L$ und $R$, so dass $R$ in Zeilenstufenform ist und
\[ M = L \cdot R \]
gilt. Schreibe $L$ als Produkt von Elementarmatrizen!
\end{frage}

\begin{frage}
Bestimme die L"osungsmenge f"ur das Gleichungssystem
\begin{align*}
x + 2 z &= 0\\
2x + y + 5z &= 0\\
3x + y + 7z &= 0.
\end{align*}
\end{frage}

\begin{frage}
Untersuche jeweils für die gegebene Matrix $A$, ob das Gleichungssystem
$A \cdot x = y$ für alle $y \in \R^3$ lösbar ist.
\begin{multicols}{3}
\begin{enumerate}[(i)]
\item $\begin{pmatrix} 1 & 3 & -1 \\ 1 & 5 & 5 \\ 2 & 7 & 1 \end{pmatrix}$
\item $\begin{pmatrix} 2 & -1 & 3 & 4 \\ 4 & -2 & 6 & 7 \\ -4 & 4 & -6 & -12 \end{pmatrix}$
\item $\begin{pmatrix} 1 & 2 \\ 3 & 4 \\ 5 & 6 \end{pmatrix}$
\end{enumerate}
\end{multicols}
\end{frage}

\begin{frage}
Untersuche jede der folgenden Matrizen auf Invertierbarkeit und bestimmte
gegebenenfalls die inverse Matrix.
\begin{multicols}{3}
\begin{enumerate}[(i)]
\item $\begin{pmatrix} 5 & 3 \\ 3 & 2 \end{pmatrix}$
\item $\begin{pmatrix} 1 & -2 & 3 \\ 2 & 1 & -4 \\ 1 & -7 & 13\end{pmatrix}$
\item $\begin{pmatrix} 2 & -1 & 3 \\ 7 & 3 &  0 \\ -1 & 2 & -4\end{pmatrix}$
\item $\begin{pmatrix} 3 & 4 \\ -1 & 2 \end{pmatrix}$
\item $\begin{pmatrix} 2 & 3 & -4 \\ 3 & 3 & -1 \\ 0 & 3 & -10\end{pmatrix}$
\item $\begin{pmatrix} 0 & 0 & 1 & 0 \\ 0 & -1 & 0 & -3 \\ 1 & 2 & 0 & 6 \\ 0 & 0 & 0 & 1 \end{pmatrix}$
\end{enumerate}
\end{multicols}
\end{frage}

\begin{frage} Sei $A \in \Mat(K, n, n)$ invertierbar.
Beweise
\[ \left(A^T \right)^{-1} = \left(A^{-1} \right)^T. \]
\emph{Hinweis:} F"ur alle multiplizierbaren Matrizen $A,B$ gilt
$(AB)^T = B^T A^T$.
\end{frage}

\begin{frage}
Untersuche folgende Abbildungen(!) auf Injektivit"at, Surjektivit"at und Bijektivit"at.
\begin{enumerate}[(i)]
\item $\phi\colon \R^2 \to \R^2, \quad \mat{x_1 \\ x_2} \mapsto \mat{1 & 2\\3 & 6} \mat{x_1 \\ x_2}$
\item $\phi\colon \R^3 \to \R^3, \quad \mat{x_1 \\ x_2 \\ x_3} \mapsto \mat{1 & 0 & 1\\0 & 1 & 1\\0 & 0 & 2} \mat{x_1 \\ x_2 \\ x_3} + \mat{1 \\ 0 \\ 1}$
\item $\phi\colon \R^m \to \R^m, \quad x \mapsto A x + b$\\ f"ur eine invertierbare Matrix $A \in \Mat(\R, m,m)$ und ein beliebiges $b \in \R^m$.
Kannst du mit diesem Aufgabenteil die Aufgabenteile (i) und (ii) schneller l"osen?
\end{enumerate}
\end{frage}

\section{Lineare Abbildungen}

\begin{frage}Seien $V$ und $W$ $K$-Vektorräume. 
Wann heißt eine Abbildung $f\colon V \to W$ linear?
\end{frage}

\clearpage

\begin{frage}
Bestimme, ob folgende Abbildungen linear sind. Sind sie affin? Bestimme gegebenenfalls eine Matrix $A$ und einen Vektor $b$, für die $f(x) = Ax + b$ gilt.
\begin{multicols}{2}
\begin{enumerate}[(i)]
\item $f\colon \R \to \R, \: x \mapsto 2x - 3$
\item $f\colon \R^3 \to \R, \: \mat{x \\ y \\ z} \mapsto xy - z$
\item $f\colon \R^3 \to \R^2, \: \mat{x \\ y \\ z} \mapsto \mat{ x -y \\ 1 + z} $
\item $f\colon \R^3 \to \R^3, \: \mat{x \\ y \\ z} \mapsto \mat{x + 2z \\ 2x + y + 5z \\ 3x + y  + 7z}$
\end{enumerate}
\end{multicols}
\end{frage}

\begin{frage}
Seien $V, W$ wieder $K$-Vektorräume und sei $f\colon V \to W$ linear. Wie sind Kern $\ker f$ und Bild $\Im f$ von $f$ defininert? Beweise, dass sie Untervektorräume sind von $V$ bzw. $W$ sind.
\end{frage}

\begin{frage} Sei $f\colon V \to W$ linear.
Beweise folgende Aussagen:
\begin{enumerate}[(i)]
\item Wenn $f$ injektiv ist, so gilt $\ker f = \{ 0 \}$
\item Sei $f(x) = Ax$ für eine Matrix $A$. Wenn $f$ surjektiv ist, so hat $A$ vollen Rang.
\item Eine lineare Abbildung $f\colon \R^n \to \R^n$ ist genau dann injektiv, wenn sie surjektiv ist.
\end{enumerate}
\end{frage}

\begin{frage}
Sei $f\colon V \to W$ linear, $f(v) = Av$ für eine Matrix $A$ und $b \in W$. Wie hängen Kern $\ker f$ und Urbild $f^{-1}(b)$ von $b$ zusammen? 
\end{frage}

\end{document}

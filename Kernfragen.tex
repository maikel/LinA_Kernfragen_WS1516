\documentclass[11pt]{scrartcl}

% Sprache und Schrift
\usepackage[utf8]{inputenc}
\usepackage[ngerman]{babel}
\usepackage{amsthm}
\usepackage{libertine} \usepackage[libertine]{newtxmath}
% \usepackage{mathspec}

% Mathematik
% \usepackage{amsthm}

% bessere Listen
\usepackage{paralist}

% eigene Umgebungen
\theoremstyle{definition}
\newtheorem{frage}{Frage}

% mathematische Aliase
\newcommand{\R}{\mathbb R}
\newcommand{\N}{\mathbb N}
\newcommand{\C}{\mathbb C}
\newcommand{\Q}{\mathbb Q}
\newcommand{\Z}{\mathbb Z}
\newcommand{\sprod}[2]{\left\langle #1, #2 \right\rangle}
\DeclareMathOperator{\lin}{span}
\DeclareMathOperator{\Mat}{Mat}

\begin{document}
{\raggedleft Autor: M. Nadolski \hfill Wintersemester 2015/2016}\\
E-Mail: nadolski@fu-berlin.de

\begin{center}
\LARGE\textbf {Kernfragen}\\
\large ``Lineare Algebra fur Ingenieurwissenschaften''
\end{center}

\section {Vektorräume}

Im Folgenden sei $K$ ein beliebiger Körper und $V$ ein $K$-Vektorraum.

\begin{frage}
Was ist ein Körper?
\end{frage}

\begin{frage}
Wie ist ein $K$-Vektorraum $(V, +, \cdot)$ definiert?
\end{frage}

\begin{frage}
Wie sind die komplexen Zahlen $\C$ definiert? Wie sind $z_1 \cdot z_2$
und $z_1 + z_2$ für komplexe Zahlen $z_1, z_2 \in \C$ definiert.
\end{frage}

\begin{frage}
Sei $z \in \C$ und $\bar z$ die konjugiert komplexe Zahl zu $z$. Wie ist
$\bar z$ definiert? Bestimmte Real- und Imaginärteil folgender Ausdrücke:
\begin{enumerate}[(i)]
    \item $z \bar z$
    \item $\frac 12 ( z + \bar z )$
    \item $\frac 12 ( z - \bar z )$
    \item $z = \frac 1 i$
    \item $z = \frac 1 {1 - i \sqrt 3}$
    \item $z = \frac {(-2 + 5i) \cdot (1 + 3i)}{2 + 3i} 
             - \left( \frac 2 {13} - \frac 3 {13} i \right)$
    \item $z = e^{i \phi}, \phi \in \R$
\end{enumerate}
\end{frage}

\begin{frage}
Sei $z \in \C$. Bestimme alle Lösungen von:
\begin{enumerate}[(i)]
    \item $z^2 = 1$
    \item $z^2 = -1$
    \item $z^2 + (1 + i) z + i = 0$
    \item $z^3 = -i$
    \item $z^4 = -4$
\end{enumerate}
\end{frage}

\begin{frage}
Wie sind Untervektorräume eines $K$-Vektorraums $V$ definiert?
\end{frage}

\begin{frage}
Sei $U \subset V$ eine Menge. Wie ist die lineare H"ulle $\lin U$ definiert?
\end{frage}

\begin{frage}
Sei $U \subset V$ eine Menge. Wann ist $U$ ein Erzeugendensystem von $V$?
\end{frage}

\begin{frage}
Seien $v_1, \ldots, v_n \in V$. Wann heißen sie linear unabhängig?
Sei $U \subset V$ eine beliebige Menge. Wann heißt $U$ linear unabhängig?
\end{frage}

\begin{frage}
Gebe zwei verschiedene aber äquivalente Definitionen einer Basis von $V$ 
an. Wie ist die Dimension $\dim_K V$ von $V$ definiert?
\end{frage}

\begin{frage}
Bestimme die Dimensionen folgender Vektorr"aume:
\begin{enumerate}[(i)]
    \item $\dim_\R \R^2$
    \item $\dim_\R \C$
    \item $\dim_\C \C$
    \item $\dim_\Q \R$
\end{enumerate}
\end{frage}

Von nun an sei $V$ ein endlich-dimensionaler Vektorraum

\begin{frage}
Seien $U,W$ Untervektorräume von $V$. Wie sind Summe $U + W$ und direkte
Summe $U \oplus W$ definiert.
\end{frage}

\begin{frage}
Beweise oder Widerlege: Ist $B_U$  eine Basis von $U$ und $B_W$ eine Basis 
von $W$, dann ist $B_U \cup B_W$ eine Basis von $U + W$.
\end{frage}

\begin{frage}
Bestimme die Dimension des Unterraums von $\R^4$, der von den Vektoren
\[
    \begin{pmatrix} 1 \\ -4 \\ -2 \\ 1 \end{pmatrix}, \:
    \begin{pmatrix} 1 \\ -3 \\ -1 \\ 2 \end{pmatrix}, \:
    \begin{pmatrix} 3 \\ -8 \\ -2 \\ 7 \end{pmatrix}
\]
aufgespannt wird.
\end{frage}

\begin{frage}
Gegeben Sei
\[
    B = \left\{
        \begin{pmatrix} 1 \\ 1 \\ 1 \end{pmatrix}, \:
        \begin{pmatrix} 1 \\ 1 \\ 0 \end{pmatrix}, \:
        \begin{pmatrix} 1 \\ 0 \\ 0 \end{pmatrix}
    \right\}
\]
Zeige, dass $B$ eine Basis des $\R^3$ ist!
\end{frage}

\begin{frage}
Gegeben Sei eine Menge $M$ von Vektoren des $\R^3$:
\[
    M = \left\{
        \begin{pmatrix} 1 \\ 2 \\ 3 \end{pmatrix}, \:
        \begin{pmatrix} 0 \\ 1 \\ 1 \end{pmatrix}, \:
        \begin{pmatrix} 2 \\ 5 \\ 7 \end{pmatrix}
    \right\}
\]
Welche Dimension hat $\lin M$? Gebe eine Basis von $\lin M$ an!
\end{frage}

\begin{frage}
Gegeben sei eine Menge $M$ von Vektoren des $\R^3$:
\[
    M = \left\{ 
        \begin{pmatrix} 1 \\ 2 \\ 3 \end{pmatrix}, \:
        \begin{pmatrix} 0 \\ 1 \\ 1 \end{pmatrix}
    \right\}
\]
Ergänze $M$ zu einer Basis von $\R^3$!
\end{frage}

\section {Matrizen und Gauß'sches Eliminationsverfahren}

Von hier an bezeichnen wir mit $\Mat(K, n, m)$ die Menge aller
$n \times m$-Matrizen mit Eintr"agen aus dem K"orper $K$.

\begin{frage}
Wie sind Matrizenaddition und -multiplikation definiert? Ist
$(\Mat(K, n, n), +, \cdot)$ ein K"orper?
\end{frage}

\begin{frage}
Wie ist der Spalten- bzw. Zeilenrang einer Matrix $M \in \Mat(\R, n, m)$
definiert?
\end{frage}

\end{document}